\begin{resumen}
	La úlcera del pie diabético (UPD) es una herida crónica presentada por el 85\% de los pacientes con diabetes mellitus. Los especialistas necesitan un método objetivo para la evaluación de las heridas para decidir si el tratamiento actual es adecuado o requiere modificaciones. La medición precisa es una tarea importante en el tratamiento de heridas crónicas, porque los cambios en los parámetros físicos de la herida son indicadores del progreso de curación. En este trabajo se explora la viabilidad de utilizar las cámaras RGB-D Intel\textregistered~RealSense\texttrademark~para detectar, segmentar, reconstruir y medir heridas crónicas en 3D. La herida se detecta de forma semi-automática, seleccionando una región de interés. El procedimiento de segmentación de heridas  propone un conjunto de redes neuronales convolucionales funcionando de ensemble para segmentar la zona de la UPD . Se utiliza el framework \textit{Open3D} para la reconstrucción de la herida, integrando la escena con el resultado de la segmentación. El sistema resultante proporciona un modelo 3D en color preciso de la herida segmentada y permite al usuario obtener estimaciones del volumen, el área y el perímetro de la herida, lo que ayuda en la selección de una terapia adecuada.
\end{resumen}

\begin{abstract}
	Resumen en inglés
\end{abstract}