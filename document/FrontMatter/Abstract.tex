\begin{resumen}
	La úlcera del pie diabético (UPD) es una herida crónica presentada por el 85\% de los pacientes con diabetes mellitus. Los especialistas necesitan un método efectiva para la evaluación de las heridas para decidir si el tratamiento actual es adecuado o requiere modificaciones. La medición precisa es una tarea importante en el tratamiento de heridas crónicas, porque los cambios en los parámetros físicos de la herida son indicadores del progreso de curación. En este trabajo se explora la viabilidad de utilizar las cámaras RGB-D Intel\textregistered~RealSense\texttrademark~para detectar, segmentar, reconstruir y medir heridas crónicas en 3D. La herida se detecta de forma semiautomática, seleccionando una región de interés. El procedimiento de segmentación de heridas  propone un conjunto de redes neuronales convolucionales funcionando de ensemble para segmentar la zona de la UPD . Se utiliza el framework \textit{Open3D} para la reconstrucción de la herida, integrando la escena con el resultado de la segmentación. El sistema resultante proporciona un modelo 3D preciso y a color de la herida segmentada y permite al usuario obtener estimaciones del volumen, el área y el perímetro de la herida, lo que ayuda en la selección de una terapia adecuada.
\end{resumen}

\begin{abstract}
Diabetic Foot Ulcer (DFU) is a chronic wound presented by 85\% of patients with diabetes mellitus. Specialists need an objective method for wound evaluation to decide if the current treatment is adequate or requires modifications. Accurate measurement is an important task in the treatment of chronic wounds, because the changes in the physical parameters of the wound are indicators of the healing progress. This work explores the feasibility of using Intel\textregistered~RealSense\texttrademark~RGB-D cameras to detect, segment, reconstruct and measure chronic wounds in 3D. The wound is detected semiautomatically after selecting a region of interest. The wound segmentation procedure proposes a set of convolutional neural networks working as an ensemble to segment the area of   the DFU. The \textit{Open3D} framework is used for wound reconstruction, integrating the scene with the segmentation result. The system provides an accurate 3D color model of the segmented wound and allows the user to determine the perimeter, area and volume of the wound, aiding in the selection of an appropriate therapy.
\end{abstract}