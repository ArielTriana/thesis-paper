\begin{opinion}

La Diabetes Mellitus es una de las cuatro principales enfermedades crónicas no transmisibles que ocupan las agendas de salud en todo el planeta. Entre las múltiples afectaciones a la salud humana que provoca se encuentran las úlceras en las extremidades inferiores que pueden ocurrir en el 15-35\% de la población diabética en algún momento de su vida. Estas úlceras, conocidas como úlceras de pie diabético (UPD), son causantes del 80\% de las amputaciones por causas no traumáticas de extremidades inferiores. Las UPD son reconocidas como un serio y costoso problema de salud.
    
Uno de los pasos esenciales para el tratamiento efectivo de las UPD es la evaluación progresiva de su severidad. Para esto se han desarrollado diferentes métodos. Casi todos incluyen entre los criterios a evaluar la medición del área y el volumen de la UPD. Evidentemente disponer de un método eficiente y no invasivo para estas mediciones es fundamental para la evaluación del tratamiento de las UPD. El sistema de salud cubano actualmente carece de un sistema efectivo para la medición de las UPD, aquí radica la extrema relevancia del presente trabajo de diploma. 
    
Ariel Alfonso Triana Pérez y Javier Alejandro Valdés González acometieron esta propuesta investigativa con entusiasmo. La aproximación inicial de ambos a la misma estuvo lejos del idílico \textquotedblleft trabajo de mesa\textquotedblright . Tuvieron que asistir con regularidad durante al menos dos meses a la consulta y la sala del Instituto de Angiología y Cirugía Cardiovascular para tomar videos con cámaras RGB-D de las UPD de pacientes tratados en dicha institución. De forma simultánea fueron estudiando temas avanzados del procesamiento de imágenes y la visión computacional mientras intentaban lograr el manejo eficiente de cámaras de profundidad, en específico la Intel\textregistered~RealSense\texttrademark~D435i. Esto lo lograron con suma independencia y todo el rigor necesario. Lograron añadir a su bagaje científico conceptos como stereo vision, depth sensing, object tracking, 3D reconstruction, image segmentation y mesh processing entre muchos otros.

El resultado satisfactorio del trabajo de Ariel y Javier, a pesar del desafío impuesto, era de esperarse dada la adición de rigor investigativo con perseverancia en la solución de problemas mostrada por ambos. El prototipo logrado es una excelente primera aproximación al desarrollo de un sistema eficiente de medición 3D de UPD para el sistema cubano de salud. Es más de lo que pensé era alcanzable en el corto plazo concedido para la investigación. Considero que Ariel Alfonso Triana Pérez y Javier Alejandro Valdés González muestran con amplitud la calidad que se espera de un egresado de Ciencia de la Computación de nuestra Facultad. Recomiendo, con suma satisfacción basada tanto en la calidad del sistema propuesto como del trabajo escrito, que se les conceda a ambos la máxima calificación en su ejercicio de culminación de estudios.

\vspace{1cm}

\begin{flushleft}
Dr. José Alejandro Mesejo Chiong

La Habana, 27 de noviembre de 2022.
\end{flushleft} 


\end{opinion}