\chapter*{Introducción}\label{chapter:introduction}
\addcontentsline{toc}{chapter}{Introducción}

La diabetes mellitus se define, según la Asociación Americana de Diabetes (ADA, por sus siglas en inglés), como una enfermedad metabólica caracterizada por hiperglucemia crónica, resultado de un déficit de la hormona insulina, la resistencia a esta o de ambas causas. Según \textit{Diabetes Atlas} en el año 2021, uno de cada siete adultos vivía con diabetes en América del Norte y el Caribe, mientras que uno de cada diez adultos vivía con la enfermedad en América Central y del Sur \cite{diabetesAtlas}. Por otra parte, según el \textit{Anuario Estadístico de Salud de Cuba}, la prevalencia de esta enfermedad en el país al finalizar el 2020, era de 66.9 por cada 1000 habitantes \cite{anuario2020}, con un incremento del 2.6 con respecto al año 2018 \cite{anuario2018}.

La diabetes mellitus es la enfermedad endocrina incurable más extendida y ocupa el tercer lugar entre las dolencias más serias que enfrenta la humanidad. El sistema cardiovascular suele afectarse precozmente en el diabético, causando infartos silenciosos, miocardiopatía diabética e hipertensión. Este padecimiento afecta en su proceso degenerativo diferentes estructuras oculares, causando retinopatía diabética la cual es el principal desencadenante de ceguera. La diabetes tiene manifestaciones renales infeccionas (pielonefritis y cistitis) y no infecciosas (nefropatía diabética), además daña el sistema nervioso periférico causando neuropatía diabética, parálisis de los nervios craneales y coma hiperosmolar diabético. La hiperglucemia en el diabético produce diversas alteraciones cutáneas, son frecuentes las infecciones por diversos gérmenes y hongos. El daño de los pequeños y grandes vasos produce atrofia de la piel que junto con la neuropatía, constituyen el terreno adecuado para la formación de úlceras en los pies. El seguimiento de esta enfermedad constituye uno de los temas de salud más importantes, debido a su prevalencia, consecuencias físicas y psicosociales \cite{roca}.

En medicina, se considera que el pie diabético son todas las lesiones que las personas diabéticas presentan en los pies, por debajo del maléolo~\footnote{Protuberancias con forma vagamente semiesférica que anatómicamente forman parte de la articulación del tobillo.}, causadas por disímiles desórdenes. Se conoce que el 85 \% de los pacientes que sufren amputaciones han padecido una úlcera del pie diabético \footnote{Llaga o herida abierta que generalmente se produce en la planta del pie.} (UPD). La duración de las úlceras puede variar desde pocas semanas hasta incluso años. Los diabéticos enfrentan un notable detrimento de su calidad de vida. 

En la actualidad los médicos cubanos utilizan varios protocolos de tratamientos para curar las lesiones. El principal tratamiento empleado es el Heberprot-P~\textregistered \cite{berlanga2013heberprot} un producto que favorece la cicatrización de la UPD y reduce el riesgo de amputación. Hasta la fecha, cerca de 20 000 casos con UPD han recibido sus beneficios como parte del ``Programa de Atención Integral al Paciente con UPD con el uso del Heberprot-P'' \cite{gonzalez2015resultados}. Otro de los tratamientos utilizado es la ozonoterapia que presenta resultados favorables para el 75 \% de los pacientes~\cite{alvarez2014beneficios}, además esta medida terapéutica ha sido combinada con el uso del Heberprot-P~\textregistered arrojando mejores resultados que los tratamientos por separado~\cite{martinez2019evolucion}. A pesar de todos estos tratamientos y estudios, los galenos cubanos no cuentan con una herramienta cuantitativa efectiva que valore la severidad y el proceso de curación de la UPD. La evaluación clínica depende en gran medida de las habilidades del experto para determinar un criterio acertado sobre la evolución del paciente. Actualmente, la evaluación clínica se establece según la opinión de tres especialistas que examinan al paciente y toman decisiones en conjunto. Bajo estas circunstancias, la aplicación de un diagnóstico y una terapia adecuada se ven retrasadas.

En base a esta problemática se han realizado numerosos trabajos investigativos dirigidos al desarrollo de métodos y aplicaciones para la medición precisa de la UPD. Una de las formas adoptadas para la medición, es la toma de imágenes fotográficas de la lesión, por ejemplo utilizando una cámara acoplada al FrameHeber 03 \cite{cabal2019quantitative}, un dispositivo desarrollado en el Centro de Ingeniería Genética y Biotecnología que ya no se encuentra en uso o producción. El FrameHeber 03 tuvo entre sus inconvenientes que debía esterilizarse cada vez que se empleaba, lo que retrasaba el flujo de pacientes en las consultas. 

Se tienen registros de trabajos que luego de la toma de las imágenes fotográficas, aplicaban distintas técnicas de segmentación de la zona donde se presenta la úlcera \cite{garcia2019mejoramiento, pena2016segmentacion, heras2022diabetic, ching2022segm3d}. En \cite{garcia2019mejoramiento} se concluye que se puede incluir el algoritmo de mezclas gaussianas para la segmentación automática de la DFU en un software para realizar las mediciones. En  \cite{pena2016segmentacion} se propone un sistema semiautomático que utiliza transformaciones de espacios de color para la segmentación. Heras en \cite{heras2022diabetic} desarrolló el sistema \emph{DFUlcer-UH: Annotation Tools}, que incluye un módulo de segmentación de los bordes de la úlcera utilizando una regresión logística y operadores morfológicos, además incluye segmentación de los tejidos internos de la úlcera utilizando el algoritmo \textit{Density Based Spatial Clustering of Applications with Noise} (DBSCAN). Mientras que en \cite{ching2022segm3d} se propone una red neuronal entrenada para la segmentación de los bordes de la UPD.

La evolución de la UPD por lo general, ocurre con la granulación de la lesión en primer lugar, lo cual conlleva un cambio de volumen, seguido por la cicatrización que se traduce en un decremento gradual del área y el perímetro de la lesión  \cite{kecelj2007measurement}. Se han desarrollado trabajos que permiten medir el área y el perímetro de la lesión con técnicas invasivas como la planimetría, sea digital o mecánica \cite{oien2002measuring}, donde se coloca un papel cuadriculado sobre la úlcera y se marca el borde para luego realizar las mediciones. Para la estimación del volumen se conocen dos técnicas invasivas que consisten en rellenar la cavidad con solución salina o con pasta de silicio y luego calcular el volumen de los materiales \cite{langemo2008measuring}. Estos métodos invasivos pueden provocar infecciones en los pacientes, incitando un retroceso en el proceso de sanación. Dentro de este contexto la reconstrucción 3D y el procesamiento de imágenes brindan opciones no invasivas. La reconstrucción 3D consiste en la representación de la escena mediante mallas poligonales utilizando información de profundidad y de color.

Se han desarrollado dos acercamientos principales a la modelación y reconstrucción 3D de las úlceras: la visión pasiva y activa \cite{zenteno2018volumetric}. La visión pasiva es aquella que directamente detecta la luz presente en la escena sin necesidad de utilizar fuentes luminosas externas \cite{malian2004medphos, plassmann1998mavis}. Por otra parte, la visión activa emplea una fuente invisible infrarroja para iluminar la escena (i.e. técnica de luces estructuradas \cite{filko2018wound, ching2022segm3d, ozturk1996new} o escáneres láseres \cite{callieri2003derma, krouskop2002noncontact}). Con la creciente prevalencia de los teléfonos inteligentes y tablets con alta resolución se han presentado trabajos con atractivas opciones para la evaluación de las UPD \cite{foltynski2014new, wang2014smartphone}. Sin embargo, muchas de estas propuestas tienen un alto precio por lo que dejan de ser factibles para los hospitales cubanos y/o son complejos de utilizar para el tratamiento rutinario en centros hospitalarios.

\section*{Objetivos}

Nuestro objetivo es diseñar un sistema de bajo costo, práctico y que no sea invasivo para el paciente, que permita tomar mediciones de las úlceras utilizando las cámaras de profundidad Intel~\textregistered RealSense~\texttrademark D435i y técnicas de reconstrucción 3D. El sistema debe medir perímetro y área (como indicadores de la cicatrización de la lesión), y volumen (como estimador de la granulación). Las mediciones deben ser estables en el tiempo. Esta tesis propone un sistema que automatiza la segmentación 2D, reconstrucción 3D y medición de las UPD a partir de imágenes RGB-D. Dicho proceso funciona de la siguiente forma: 

\begin{enumerate}
	\item Para la captura de la secuencia se utiliza una cámara RGB-D~\footnote{Una cámara RGB–D es una cámara que capta imágenes a colores (RGB - formato \textit{Red} (rojo), \textit{Green} (verde), \textit{Blue} (azul)) y tiene sensor de profundidad (D - \textit{depth} (profundidad)).}.
	\item Se posprocesa la secuencia mejorando el contraste de las imágenes, aplicando filtros de enfoque y otras operaciones.
	\item Luego, se emplean 10 Redes Neuronales Convolucionales, funcionando en conjunto utilizando el promedio, se segmentan las imágenes RGB que resultan en máscaras binarias.
	\item Posteriormente, se aplican operadores morfológicos y otras operaciones a las máscaras para eliminar ruido resultante de la predicción.
	\item Antes de medir, se procede a realizar la reconstrucción 3D utilizando las máscaras.
\end{enumerate}


En consecuencia, el modelo 3D final consiste en una malla que representa de la forma más precisa posible solo la superficie de la región que ocupa la UPD. Una vez obtenido el modelo 3D de la úlcera, se procede a realizar las estimaciones de perímetro, área y volumen.

Para lograr el objetivo general se proponen objetivos específicos mencionados a continuación:

\begin{enumerate}
	\item Creación de un conjunto de muestras de UPD que contenga vídeos grabados con la cámara RGB-D para la validación de la segmentación, reconstrucción y medición de las lesiones.
	\item Elaborar el sistema de segmentación, reconstrucción 3D y medición de la herida con del uso de la cámara RGB-D. Aplicar técnicas de pre/posprocesamiento para mejorar la calidad de los resultados.
	\item Analizar el desempeño de los resultados obtenidos respecto a la segmentación manual del conjunto de datos.
\end{enumerate}

Para cada uno de los propósitos específicos de este tema se combinarán una serie de técnicas y algoritmos cuidadosamente seleccionados de la previa revisión bibliográfica. Lográndose así, un nuevo sistema de segmentación, reconstrucción 3D y medición con el uso de una cámara RGB–D Intel~\textregistered RealSense™ D435i.

\section*{Organización de los Capítulos}

Esta tesis cuenta con un total de 4 Capítulos. En el Capítulo 1 se expone la revisión de la literatura donde se presentan algunos de los algoritmos existentes en las problemáticas descritas. El Capítulo 2 explica de forma breve los fundamentos teóricos, necesarios para la comprensión del trabajo. El Capítulo 3 da a conocer la estructura general de la propuesta final y una explicación de la selección de los algoritmos en las diferentes fases. Luego, en el Capítulo 4 se presentan los resultados obtenidos y se comparan con los de otros sistemas. Posteriormente, se dan las conclusiones y recomendaciones de la investigación con el propósito de su continuación. Para finalizar, se adjunta la bibliografía del trabajo.