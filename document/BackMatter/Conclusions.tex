\begin{conclusions}

La creación de un sistema de bajo costo y no intrusivo para llevar a cabo mediciones en las UPDs es de vital importancia, pues permite brindar a los centros médicos una herramienta para el seguimiento de la evolución de los tratamientos aplicados a dicha enfermedad. En particular, esta investigación propone un sistema de reconstrucción, segmentación 3D y mediciones de perímetro, superficie y volúmen de UPDs a partir de la captura de una secuencia de imágenes RGB-D. Esto conforma el paso base para la realización de un software intuitivo que pueda ser manejado directamente por los especialistas de centros médicos cubanos.

La implementación consiste en varias etapas. Primeramente el usuario selecciona una región de interés con forma rectangular para luego comenzar con la captura de la secuencia, aplicando filtros de pre/post-procesamiento a los mapas de color y profundidad respectivamente. A continuación se procede a segmentar el conjunto de imágenes de color, usando un ensemble entre las redes neuronales LinkNet y UNet, obteniendo las máscaras binarias. Se utiliza la secuencia de la captura para la creación de fragmentos y su registro global, para luego integrar la escena usando las imágenes segmentadas. Como resultado, se obtiene una malla de la cual solo tomamos la región con color. Por último se procede a tomar mediciones a la malla resultante.

Como resultado final, se alcanzó un error absoluto de aproximadamente 10mm para el cálculo de perímetro, mientras que para el área y volumen se alcanzaron valores de $470mm^2$ y $3289mm^3$. Se observa como en medidas de mayor dimensión tienden a aumentar los errores, sin embargo, los resultados obtenidos lucen prometedores ante trabajos futuros. A continuación, se proponen un conjunto de recomendaciones y mejoras para posteriores investigaciones, esperando la satisfactoria continuidad de este proyecto.
\end{conclusions}
